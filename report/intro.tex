
With the progress of the times, email has been one of the most efficient communication media nowadays. However, with the ubiquitous use of email, useless messages and advertisements spread widely which caused email misuse. To improve this issue, lots of people did the researche for some approaches to construct filters by using machine learning. Most researche mainly focus on adjusting classical methods to make filters more efficient.\\

To have better understanding, we referred from lots of papers about detection of spam email. Some papers tried to compare different methods and found the best one. However, the results in different papers sometimes are different because most of them did not compare the same methods at the same time. Another reason may come from different data set they used. Moreover, the data set they used typically are experimental and smaller. Therefore, we want to make a overall comparison for the methods they tried and apply those methods on larger data set we combined. We would compare four methods: SVMs, Decision Trees, Naive Bayesian, KNN, Multi Layer Perceptron (MLP), and Logistics by calculating accuracy rate and cost time. In addition, we would use two kinds of data prepossessing (unit-gram and bi-gram tfidf) to increase complexity in our project.\\

Moreover, our data set contains over than 150,000 email from 1999 to 2007. We supposed that the keyword per year would change because of the innovated technology or social cognition. We would like to discuss keywords in spam email by year to explore the characteristics in each year.


